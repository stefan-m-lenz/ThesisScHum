\HeaderA{JuliaConnectoR-package}{A Functionally Oriented Interface for Integrating Julia with R}{JuliaConnectoR.Rdash.package}
%
\begin{Description}\relax
This package provides a functionally oriented interface between R and Julia.
The goal is to call functions from Julia packages directly as R functions.
\end{Description}
%
\begin{Details}\relax
This R-package provides a functionally oriented interface between R and Julia.
The goal is to call functions from Julia packages directly as R functions.
Julia functions imported via the \pkg{JuliaConnectoR} can accept and return R variables.
It is also possible to pass R functions as arguments in place of Julia functions,
which allows \emph{callbacks} from Julia to R.

From a technical perspective, R data structures are serialized with an optimized custom streaming format,
sent to a (local) Julia TCP server, and translated to Julia data structures by Julia.
The results are returned back to R.
Simple objects, which correspond to vectors in R, are directly translated.
Complex Julia structures are by default transferred to R by reference via proxy objects.
This enables an effective and intuitive handling of the Julia objects via R.
It is also possible to fully translate Julia objects to R objects.
These translated objects are annotated with information
about the original Julia objects, such that they can be translated back to Julia.
This makes it also possible to serialize them as R objects.
\end{Details}
%
\begin{Section}{Setup}

The package requires that
\Rhref{https://julialang.org/downloads/}{Julia (Version \eqn{\geq}{} 1.0) is installed}
and that the Julia executable is in the system search \env{PATH} or that the
\env{JULIA\_BINDIR} environment variable is set to the \code{bin} directory of
the Julia installation.
\end{Section}
%
\begin{Section}{Function overview}

The function \code{\LinkA{juliaImport}{juliaImport}} makes
functions and data types from Julia packages or modules available as R functions.

If only a single Julia function needs to be importedR, \code{\LinkA{juliaFun}{juliaFun}}
can do this. The simplest way to call a Julia function without any importing
is to use \code{\LinkA{juliaCall}{juliaCall}} with the function name given
as character string.

For evaluating expressions in Julia, \code{\LinkA{juliaEval}{juliaEval}} and
\code{\LinkA{juliaLet}{juliaLet}} can be used. With \code{\LinkA{juliaLet}{juliaLet}} one can use
R variables in a expression.

\code{\LinkA{juliaExpr}{juliaExpr}} makes it possible use complex Julia syntax in R via R strings
that contain Julia expressions.

With \code{\LinkA{juliaGet}{juliaGet}}, a full translation of a Julia proxy object into an R object
is performed.

\code{as.data.frame} is overloaded (\code{\LinkA{as.data.frame.JuliaProxy}{as.data.frame.JuliaProxy}})
for translating Julia objects that implement the
\Rhref{https://github.com/JuliaData/Tables.jl}{\code{Tables}} interface
to R data frames.
\end{Section}
%
\begin{Section}{Translation}


Since Julia is more type-sensitive than R, and many Julia functions expect to be called
using specific types, it is important to know the translations of the R data structures
to Julia.

%
\begin{SubSection}{Translation from R to Julia}
The type correspondences of the basic R data types in Julia are the following:


\Tabular{lcl}{
\strong{R} &  & \strong{Julia}\\{}
\code{integer} & \eqn{\rightarrow}{} & \code{Int} \\{}
\code{double}  & \eqn{\rightarrow}{} & \code{Float64} \\{}
\code{logical}   & \eqn{\rightarrow}{} & \code{Bool} \\{}
\code{character} & \eqn{\rightarrow}{} & \code{String} \\{}
\code{complex} & \eqn{\rightarrow}{} & \code{Complex\{Float64\}} \\{}
\code{raw}  & \eqn{\rightarrow}{} & \code{UInt8} \\{}
\code{symbol} & \eqn{\rightarrow}{} & \code{Symbol}
}

R vectors of length 1 of the types in the table above will be translated to the types shown.

R vectors or arrays with more than one element will be translated to Julia \code{Array}s
of the corresponding types. The dimensions of an R array, as returned by \code{dim()},
will also be respected.
For example, the R integer vector \code{c(1L, 2L)} will be of type \code{Vector\{Int\}},
or \code{Array\{Int,1\}}, in Julia.
A double matrix such as \code{matrix(c(1,2,3,4), nrow = 2)}
will be of type \code{Array\{Float64,2\}}.

Missing values (\code{NA}) in R are translated to \code{missing} values in Julia.
R vectors and arrays with missing values are converted to Julia arrays
of type \code{Array\{Union\{Missing, T\}\}}, where \code{T} stands for the translated 
type in the table above.

R lists are translated as \code{Vector\{T\}} in Julia, with \code{T} being
the most specific supertype of the list elements after translation to Julia.

An R function that is handed to Julia as argument in a function
call is translated to a Julia callback function that will call the given R function.

Strings with attribute \code{"JLEXPR"}
will be evaluated as Julia expressions,
and the value is used in their place (see \code{\LinkA{juliaExpr}{juliaExpr}}).

R data frames are translated to objects that implement the Julia
\Rhref{https://github.com/JuliaData/Tables.jl}{\code{Tables}} interface.
Such objects can be used by functions of many different
Julia packages that deal with table-like data structures.

\end{SubSection}


%
\begin{SubSection}{Translation from Julia to R}
The type system of Julia is richer than that of R. Therefore, to be able to turn
the Julia data structures that have been translated to R back to the original Julia
data structures, the original Julia types are added to the translated Julia objects
in R via the attribute \code{"JLTYPE"}.
When passed to Julia, R variables with this
attribute will be coerced to the respective type.
This allows the reconstruction of the objects
with their original type.

It should not be necessary to worry too much
about the translations from Julia to R because the resulting R objects should be
intuitive to handle.

The following table shows how basic R-compatible types of Julia are translated to R:

\Tabular{lcl}{
\strong{Julia} &  & \strong{R} \\{}
\code{Float64}& \eqn{\rightarrow}{} &\code{double} \\{}
\code{Float16}, \code{Float32}, \code{UInt32} & \eqn{\rightarrow}{} &\code{double} with type attribute \\{}
\code{Int64} that fits in 32 bits & \eqn{\rightarrow}{} & \code{integer} \\{}
\code{Int64} not fitting in 32 bits & \eqn{\rightarrow}{} & \code{double} with type attribute \\{}
\code{Int8}, \code{Int16}, \code{UInt16}, \code{Int32}, \code{Char} & \eqn{\rightarrow}{} &\code{integer} with type attribute \\{}
\code{UInt8}& \eqn{\rightarrow}{} &\code{raw} \\{}
\code{UInt64}, \code{Int128}, \code{UInt128}, \code{Ptr} & \eqn{\rightarrow}{} &\code{raw} with type attribute \\{}
\code{Complex\{Float64\}}& \eqn{\rightarrow}{} &\code{complex} \\{}
\code{Complex\{Int\var{X}\}} with \var{X} \eqn{\leq}{} 64 & \eqn{\rightarrow}{} &\code{complex} with type attribute \\{}
\code{Complex\{Float\var{X}\}} with \var{X} \eqn{\leq}{} 32 & \eqn{\rightarrow}{} &\code{complex} with type attribute
}

Julia \code{Array}s of these types are translated to \code{vector}s or \code{array}s of the corresponding types in R.


Julia functions are translated to R functions that call the Julia function.
These functions can also be translated back to the
corresponding Julia functions when used as argument of another function
(see \code{\LinkA{juliaFun}{juliaFun}}).


Julia object of other types, in particular \code{struct}s, \code{Tuple}s, \code{NamedTuple}s,
and \code{AbstractArray}s of other types are transferred by reference in the form of proxy objects.
Elements and properties of these proxy objects can be accessed and mutated via the operators \code{`[[`},
\code{`[`}, and \code{`\$`} (see \LinkA{AccessMutate.JuliaProxy}{AccessMutate.JuliaProxy}).

A full translation of the proxy objects into R objects, which also allows saving these objects in R,
is possible via \code{\LinkA{juliaGet}{juliaGet}}.


\end{SubSection}

\end{Section}
%
\begin{Section}{Limitations}


Numbers of type \code{Int64} that are too big to be expressed as 32-bit
\code{integer} values in R will be translated to \code{double} numbers.
This may lead to a inaccurate results for very large numbers,
when they are translated back to Julia, since, e. g.,
\code{(2\textasciicircum{}53 + 1) - 2\textasciicircum{}53 == 0} holds for double-precision
floating point numbers.
\end{Section}
\inputencoding{utf8}
\HeaderA{AccessMutate.JuliaProxy}{Access or mutate Julia objects via proxy objects}{AccessMutate.JuliaProxy}
\aliasA{\$.JuliaStructProxy}{AccessMutate.JuliaProxy}{.Rdol..JuliaStructProxy}
\aliasA{\$<\Rdash{}.JuliaStructProxy}{AccessMutate.JuliaProxy}{.Rdol.<.Rdash..JuliaStructProxy}
\aliasA{dim.JuliaArrayProxy}{AccessMutate.JuliaProxy}{dim.JuliaArrayProxy}
\aliasA{length.JuliaArrayProxy}{AccessMutate.JuliaProxy}{length.JuliaArrayProxy}
\aliasA{[.JuliaProxy}{AccessMutate.JuliaProxy}{[.JuliaProxy}
\aliasA{[.JuliaSimpleArrayProxy}{AccessMutate.JuliaProxy}{[.JuliaSimpleArrayProxy}
\aliasA{[<\Rdash{}.JuliaProxy}{AccessMutate.JuliaProxy}{[<.Rdash..JuliaProxy}
\aliasA{[[.JuliaArrayProxy}{AccessMutate.JuliaProxy}{[[.JuliaArrayProxy}
\aliasA{[[.JuliaStructProxy}{AccessMutate.JuliaProxy}{[[.JuliaStructProxy}
\aliasA{[[<\Rdash{}.JuliaArrayProxy}{AccessMutate.JuliaProxy}{[[<.Rdash..JuliaArrayProxy}
\aliasA{[[<\Rdash{}.JuliaStructProxy}{AccessMutate.JuliaProxy}{[[<.Rdash..JuliaStructProxy}
%
\begin{Description}\relax
Apply the R operators \code{\$} and \code{\$<-}, \code{[} and \code{[<-}, \code{[[}
and \code{[[<-} to access or modify parts of Julia objects via their proxy objects.
For an intuitive understanding, best see the examples below.
\end{Description}
%
\begin{Usage}
\begin{verbatim}
## S3 method for class 'JuliaStructProxy'
x$name

## S3 replacement method for class 'JuliaStructProxy'
x$name <- value

## S3 method for class 'JuliaProxy'
x[...]

## S3 replacement method for class 'JuliaProxy'
x[i, j, k] <- value

## S3 method for class 'JuliaSimpleArrayProxy'
x[...]

## S3 method for class 'JuliaArrayProxy'
x[[...]]

## S3 replacement method for class 'JuliaArrayProxy'
x[[i, j, k]] <- value

## S3 method for class 'JuliaStructProxy'
x[[name]]

## S3 replacement method for class 'JuliaStructProxy'
x[[name]] <- value

## S3 method for class 'JuliaArrayProxy'
length(x)

## S3 method for class 'JuliaArrayProxy'
dim(x)
\end{verbatim}
\end{Usage}
%
\begin{Arguments}
\begin{ldescription}
\item[\code{x}] a Julia proxy object

\item[\code{name}] the field of a struct type, the name of a member in a \code{NamedTuple},
or a key in a Julia dictionary (type \code{AbstractDict})

\item[\code{value}] a suitable replacement value.
When replacing a range of elements in an array type, it is possible to
replace multiple elements with single elements. In all other cases,
the length of the replacement must match the number of elements to replace.

\item[\code{i, j, k, ...}] index(es) for specifying the elements to extract or replace
\end{ldescription}
\end{Arguments}
%
\begin{Details}\relax
The operators \code{\$} and \code{[[} allow to access properties of Julia \code{struct}s
and \code{NamedTuple}s via their proxy objects.
For dictionaries (Julia type \code{AbstractDict}), \code{\$} and \code{[[}
can also be used to look up string keys.
Fields of \code{mutable struct}s and dictionary elements with string keys
can be set via \code{\$<-} and \code{[[<-}.

For \code{AbstractArray}s, the \code{[}, \code{[<-}, \code{[[}, and \code{[[<-}
operators relay to the \code{getindex} and \code{setindex!} Julia functions.
The \code{[[} and \code{[[<-} operators are used to access or mutate a single element.
With \code{[} and \code{[<-}, a range of objects is accessed or mutated.
The elements of \code{Tuple}s can also be accessed via \code{[} and \code{[[}.

The dimensions of proxy objects for Julia \code{AbstractArray}s and \code{Tuple}s
can be queried via \code{length} and \code{dim}.
\end{Details}
%
\begin{Examples}
\begin{ExampleCode}
if (juliaSetupOk()) {

   # (Mutable) struct
   juliaEval("mutable struct MyStruct
                x::Int
             end")

   MyStruct <- juliaFun("MyStruct")
   s <- MyStruct(1L)
   s$x
   s$x <- 2
   s[["x"]]

   # Array
   x <- juliaCall("map", MyStruct, c(1L, 2L, 3L))
   x
   length(x)
   x[[1]]
   x[[1]]$x
   x[[1]] <- MyStruct(2L)
   x[2:3]
   x[2:3] <- MyStruct(2L)
   x

   # Tuple
   x <- juliaEval("(1, 2, 3)")
   x[[1]]
   x[1:2]
   length(x)

   # NamedTuple
   x <- juliaEval("(a=1, b=2)")
   x$a

   # Dictionary
   strDict <- juliaEval('Dict("hi" => 1, "hello" => 2)')
   strDict
   strDict$hi
   strDict$hi <- 0
   strDict[["hi"]] <- 2
   strDict["howdy", "greetings"] <- c(2, 3)
   strDict["hi", "howdy"]

}


\end{ExampleCode}
\end{Examples}
\inputencoding{utf8}
\HeaderA{as.data.frame.JuliaProxy}{Coerce a Julia Table to a Data Frame}{as.data.frame.JuliaProxy}
%
\begin{Description}\relax
Get the data from a Julia proxy object that implements the Julia
\Rhref{https://github.com/JuliaData/Tables.jl}{\code{Tables}} interface,
and create an R data frame from it.
\end{Description}
%
\begin{Usage}
\begin{verbatim}
## S3 method for class 'JuliaProxy'
as.data.frame(x, ...)
\end{verbatim}
\end{Usage}
%
\begin{Arguments}
\begin{ldescription}
\item[\code{x}] a proxy object pointing to a Julia object that implements the interface
of the package Julia package \code{Tables}

\item[\code{...}] (not used)
\end{ldescription}
\end{Arguments}
%
\begin{Details}\relax
Strings are not converted to factors.
\end{Details}
%
\begin{Examples}
\begin{ExampleCode}
if (juliaSetupOk()) {

   # Demonstrate the usage with the Julia package "JuliaDB"
   juliaEval('import Pkg; Pkg.add("JuliaDB")')
   JuliaDB <- juliaImport("JuliaDB")

   mydf <- data.frame(x = c(1, 2, 3),
                      y = c("a", "b", "c"),
                      z = c(TRUE, FALSE, NA),
                      stringsAsFactors = FALSE)

   # create a table in Julia, e. g. via JuliaDB
   mytbl <- JuliaDB$table(mydf)

   # this table can, e g. be queried and
   # the result can be translated to an R data frame
   seltbl <- JuliaDB$select(mytbl, juliaExpr("(:x, :y)"))[1:2]

   # translate selection of Julia table into R data frame
   as.data.frame(seltbl)

}


\end{ExampleCode}
\end{Examples}
\inputencoding{utf8}
\HeaderA{juliaCall}{Call a Julia function by name}{juliaCall}
%
\begin{Description}\relax
Call a Julia function via specifying the name as string and get the translated result.
It is also possible to use a dot at the end of the function name
for applying the function in a vectorized manner via "broadcasting" in Julia.
\end{Description}
%
\begin{Usage}
\begin{verbatim}
juliaCall(name, ...)
\end{verbatim}
\end{Usage}
%
\begin{Arguments}
\begin{ldescription}
\item[\code{name}] name of the Julia function. May be prefixed with a package or module
path.

\item[\code{...}] parameters handed to the function. Will be translated
to Julia data structures
\end{ldescription}
\end{Arguments}
%
\begin{Value}
The value returned from Julia, translated to an R data structure.
If Julia returns \code{nothing}, an invisible \code{NULL} is returned.
\end{Value}
%
\begin{Examples}
\begin{ExampleCode}
if (juliaSetupOk()) {

   juliaCall("/", 4, 2)
   juliaCall("Base.div", 4, 2)
   juliaCall("sin.", c(1,2,3))
   juliaCall("Base.cos.", c(1,2,3))

}


\end{ExampleCode}
\end{Examples}
\inputencoding{utf8}
\HeaderA{juliaEval}{Evaluate a Julia expression}{juliaEval}
%
\begin{Description}\relax
This function evaluates Julia code, given as a string, in Julia,
and translates the result back to R.
\end{Description}
%
\begin{Usage}
\begin{verbatim}
juliaEval(expr)
\end{verbatim}
\end{Usage}
%
\begin{Arguments}
\begin{ldescription}
\item[\code{expr}] Julia code, given as a one-element character vector
\end{ldescription}
\end{Arguments}
%
\begin{Details}\relax
If the code needs to use R variables, consider using \code{juliaLet}
instead.
\end{Details}
%
\begin{Value}
The value returned from Julia, translated to an R data structure.
If Julia returns \code{nothing}, an invisible \code{NULL} is returned.
This is also the case if the last non-whitespace character of \code{expr}
is a semicolon.
\end{Value}
%
\begin{Examples}
\begin{ExampleCode}
if (juliaSetupOk()) {

   juliaEval("1 + 2")
   juliaEval('using Pkg; Pkg.add("BoltzmannMachines")')
   juliaEval('using Random; Random.seed!(5);')

}


\end{ExampleCode}
\end{Examples}
\inputencoding{utf8}
\HeaderA{juliaExpr}{Mark a string as Julia expression}{juliaExpr}
%
\begin{Description}\relax
A given R character vector is marked as a Julia expression.
It will be executed and evaluated when passed to Julia.
This allows to pass a Julia object that is defined by complex Julia syntax
as an argument without needing the round-trip to R via \code{\LinkA{juliaEval}{juliaEval}}
or \code{\LinkA{juliaLet}{juliaLet}}.
\end{Description}
%
\begin{Usage}
\begin{verbatim}
juliaExpr(expr)
\end{verbatim}
\end{Usage}
%
\begin{Arguments}
\begin{ldescription}
\item[\code{expr}] a character vector which should contain one string
\end{ldescription}
\end{Arguments}
%
\begin{Examples}
\begin{ExampleCode}
if (juliaSetupOk()) {

   # Create complicated objects like version strings in Julia, and compare them
   v1 <- juliaExpr('v"1.0.1"')
   v2 <- juliaExpr('v"1.2.0"')
   juliaCall("<", v1, v2)

}


\end{ExampleCode}
\end{Examples}
\inputencoding{utf8}
\HeaderA{juliaFun}{Wrap a Julia function in an R function}{juliaFun}
%
\begin{Description}\relax
Creates an R function that will call the Julia function with the given name
when it is called. Like any R function, the returned function can
also be passed as a function argument to Julia functions.
\end{Description}
%
\begin{Usage}
\begin{verbatim}
juliaFun(name, ...)
\end{verbatim}
\end{Usage}
%
\begin{Arguments}
\begin{ldescription}
\item[\code{name}] the name of the Julia function

\item[\code{...}] optional arguments for currying:
The resulting function will be called using these arguments.
\end{ldescription}
\end{Arguments}
%
\begin{Examples}
\begin{ExampleCode}
if (juliaSetupOk()) {

   # Wrap a Julia function and use it
   juliaSqrt <- juliaFun("sqrt")
   juliaSqrt(2)
   # In the following call, the sqrt function is called without
   # a callback to R because the linked function object is used.
   juliaCall("map", juliaSqrt, c(1,4,9))

   # may also be used with arguments
   plus1 <- juliaFun("+", 1)
   plus1(2)
   # Results in an R callback (calling Julia again)
   # because there is no linked function object in Julia.
   juliaCall("map", plus1, c(1,2,3))

}


\end{ExampleCode}
\end{Examples}
\inputencoding{utf8}
\HeaderA{juliaGet}{Translate a Julia proxy object to an R object}{juliaGet}
%
\begin{Description}\relax
R objects of class \code{JuliaProxy} are references to Julia objects in the Julia session.
These R objects are also called "proxy objects".
With this function it is possible to translate these objects into R objects.
\end{Description}
%
\begin{Usage}
\begin{verbatim}
juliaGet(x)
\end{verbatim}
\end{Usage}
%
\begin{Arguments}
\begin{ldescription}
\item[\code{x}] a reference to a Julia object
\end{ldescription}
\end{Arguments}
%
\begin{Details}\relax
If the corresponding Julia objects do not contain external references,
translated objects can also saved in R and safely be restored in Julia.

Modifying objects is possible and changes in R will be translated back to Julia.

The following table shows the translation of Julia objects into R objects.


\Tabular{lcl}{
\strong{Julia} &  & \strong{R} \\{}
\code{struct} & \eqn{\rightarrow}{} & \code{list} with the named struct elements \\{}
\code{Array} of \code{struct} type & \eqn{\rightarrow}{} & \code{list} (of \code{list}s) \\{}
\code{Tuple} & \eqn{\rightarrow}{} & \code{list} \\{}
\code{NamedTuple} & \eqn{\rightarrow}{} & \code{list} with the named elements \\{}
\code{AbstractDict} & \eqn{\rightarrow}{} & \code{list} with two sub-lists: "\code{keys}" and "\code{values}" \\{}
\code{AbstractSet} & \eqn{\rightarrow}{} & \code{list} \\
}
\end{Details}
%
\begin{Note}\relax
Objects containing cicular references cannot be translated back to Julia.

It is safe to translate objects that contain external references from Julia to R.
The pointers will be copied as values and the finalization of the translated
Julia objects is prevented.
The original objects are garbage collected after all direct or
indirect copies are garbage collected.
Note, however, that these translated objects cannot be translated back to Julia
after the Julia process has been stopped and restarted.
\end{Note}
\inputencoding{utf8}
\HeaderA{juliaImport}{Load and import a Julia package via \code{import} statement}{juliaImport}
%
\begin{Description}\relax
The specified package/module is loaded via \code{import} in Julia.
Its functions and type constructors are wrapped into R functions.
The return value is an environment containing all these R functions.
\end{Description}
%
\begin{Usage}
\begin{verbatim}
juliaImport(modulePath, all = TRUE)
\end{verbatim}
\end{Usage}
%
\begin{Arguments}
\begin{ldescription}
\item[\code{modulePath}] a module path or a module object.
A module path may simply be the name of a package but it may also
be a relative module path.
Specifying a relative Julia module path like \code{.MyModule}
allows importing a module that does not correspond to a package,
but has been loaded in the \code{Main} module, e. g. by \\
\code{juliaCall("include", "path/to/MyModule.jl")}.
Additionally, via a path such as \code{SomePkg.SubModule},
a submodule of a package can be imported.

\item[\code{all}] \code{logical} value, default \code{TRUE}.
Specifies whether all functions and types shall be imported
or only those exported explicitly.
\end{ldescription}
\end{Arguments}
%
\begin{Value}
an environment containing all functions and type constructors
from the specified module as R functions
\end{Value}
%
\begin{Examples}
\begin{ExampleCode}
if (juliaSetupOk()) {

   # Importing a package and using one of its exported functions
   UUIDs <- juliaImport("UUIDs")
   juliaCall("string", UUIDs$uuid4())


   # Importing a module without a package
   testModule <- system.file("examples", "TestModule1.jl",
                             package = "JuliaConnectoR")
   # take a look at the file
   writeLines(readLines(testModule))
   # load in Julia
   juliaCall("include", testModule)
   # import in R via relative module path
   TestModule1 <- juliaImport(".TestModule1")
   TestModule1$test1()
   
   # Importing a local module is also possible in one line,
   # by directly using the module object returned by "include".
   TestModule1 <- juliaImport(juliaCall("include", testModule))
   TestModule1$test1()
}


if (juliaSetupOk()) {

   # Importing a submodule
   testModule <- system.file("examples", "TestModule1.jl",
                             package = "JuliaConnectoR")
   juliaCall("include", testModule)
   # load sub-module via module path
   SubModule1 <- juliaImport(".TestModule1.SubModule1")
   # call function of submodule
   SubModule1$test2()

}


\end{ExampleCode}
\end{Examples}
\inputencoding{utf8}
\HeaderA{juliaLet}{Evaluate Julia code in a \code{let} block using values of R variables}{juliaLet}
%
\begin{Description}\relax
R variables can be passed as named arguments, which are inserted
for those variables in the Julia expression that have the same name
as the named arguments. The given Julia code is executed in Julia
inside a \code{let} block and the result is translated back to R.
\end{Description}
%
\begin{Usage}
\begin{verbatim}
juliaLet(expr, ...)
\end{verbatim}
\end{Usage}
%
\begin{Arguments}
\begin{ldescription}
\item[\code{expr}] Julia code, given as one-element character vector

\item[\code{...}] arguments that will be introduced as variables in the
\code{let} block. The values are transferred to Julia and
assigned to the variables introduced in the \code{let} block.
\end{ldescription}
\end{Arguments}
%
\begin{Details}\relax
A simple, nonsensical example for explaining the principle:

\code{juliaLet('println(x)', x = 1)}

This is the same as

\code{juliaEval('let x = 1.0; println(x) end')}

More complex objects cannot be simply represented in a string like in
this simple example any more.
That is the problem that \code{juliaLet} solves.

Note that the evaluation is done in a \code{let} block. Therefore,
changes to global variables in the Julia session are only possible by
using the keyword \code{global} in front of the Julia variables
(see examples).
\end{Details}
%
\begin{Value}
The value returned from Julia, translated to an R data structure.
If Julia returns \code{nothing}, an invisible \code{NULL} is returned.
\end{Value}
%
\begin{Examples}
\begin{ExampleCode}
if (juliaSetupOk()) {

   # Intended use: Create a complex Julia object
   # using Julia syntax and data from the R workspace
   juliaLet('[1 => x, 17 => y]', x = rnorm(1), y = rnorm(2))

   # Assign a global variable
   # (although not recommended for a functional style)
   juliaLet("global x = xval", xval = rnorm(10))
   juliaEval("x")

}


\end{ExampleCode}
\end{Examples}
\inputencoding{utf8}
\HeaderA{juliaPut}{Create a Julia proxy object from an R object}{juliaPut}
%
\begin{Description}\relax
This function creates a proxy object for a Julia object that would
otherwise be translated to an R object.
This is useful to prevent many translations of large objects
if it is necessary performance reasons.
To see which objects are translated by default, please see the
\LinkA{JuliaConnectoR-package}{JuliaConnectoR.Rdash.package} documentation.
\end{Description}
%
\begin{Usage}
\begin{verbatim}
juliaPut(x)
\end{verbatim}
\end{Usage}
%
\begin{Arguments}
\begin{ldescription}
\item[\code{x}] an R object (can also be a translated Julia object)
\end{ldescription}
\end{Arguments}
%
\begin{Examples}
\begin{ExampleCode}
if (juliaSetupOk()) {

   # Transfer a large vector to Julia and use it in multiple calls
   x <- juliaPut(rnorm(100))
   # x is just a reference to a Julia vector now
   juliaEval("using Statistics")
   juliaCall("mean", x)
   juliaCall("var", x)

}


\end{ExampleCode}
\end{Examples}
\inputencoding{utf8}
\HeaderA{juliaSetupOk}{Check Julia setup}{juliaSetupOk}
%
\begin{Description}\relax
Checks that Julia can be started and that the Julia version is at least 1.0.
\end{Description}
%
\begin{Usage}
\begin{verbatim}
juliaSetupOk()
\end{verbatim}
\end{Usage}
%
\begin{Value}
\code{TRUE} if the Julia setup is OK; otherwise \code{FALSE}
\end{Value}