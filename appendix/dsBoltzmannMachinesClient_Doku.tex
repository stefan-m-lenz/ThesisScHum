\HeaderA{ds.bm.defineLayer}{Define training parameters for one RBM layer in a DBM or DBN}{ds.bm.defineLayer}
%
\begin{Description}\relax
The call stores the given parameters for training one RBM-layer in a DBM or DBN
on the server side in a Julia \code{TrainLayer} object.
The parameters \code{rbmtype}, \code{nhidden}, \code{epochs},
\code{learningrate}/\code{learningrates}, \code{categories},
\code{batchsize}, \code{pcd}, \code{cdsteps}, \code{startrbm} and \code{monitoring}
of this object are passed to \code{\LinkA{ds.monitored\_fitrbm}{ds.monitored.Rul.fitrbm}}.
For a detailed description, see there.
Values of \code{NULL} indicate that a corresponding default value should be used.
\end{Description}
%
\begin{Usage}
\begin{verbatim}
ds.bm.defineLayer(
  datasources,
  newobj,
  epochs = NULL,
  learningrate = NULL,
  learningrates = NULL,
  sdlearningrate = NULL,
  sdlearningrates = NULL,
  categories = NULL,
  monitoring = NULL,
  rbmtype = NULL,
  nhidden = NULL,
  nvisible = NULL,
  batchsize = NULL,
  pcd = NULL,
  cdsteps = NULL,
  startrbm = NULL
)
\end{verbatim}
\end{Usage}
%
\begin{Arguments}
\begin{ldescription}
\item[\code{datasources}] A list of Opal object(s) as a handle to the server-side session

\item[\code{newobj}] The name of the server-side object where the parameters are stored

\item[\code{nvisible}] The number of visible units in the RBM. Only relevant for partitioning.
This parameter is derived as much as possible by \code{\LinkA{ds.monitored\_stackrbms}{ds.monitored.Rul.stackrbms}}.
For multimodal DBMs with a partitioned first layer, it is necessary to specify
the number of visible nodes for all but at most one partition in the input layer.
\end{ldescription}
\end{Arguments}
\inputencoding{utf8}
\HeaderA{ds.bm.definePartitionedLayer}{Define parameters for training a partitioned RBM layer in a DBM or DBN}{ds.bm.definePartitionedLayer}
%
\begin{Description}\relax
Creates an object at the server-side that encapsulates the parameters for training
a partitioned layer.
\end{Description}
%
\begin{Usage}
\begin{verbatim}
ds.bm.definePartitionedLayer(datasources, newobj, parts)
\end{verbatim}
\end{Usage}
%
\begin{Arguments}
\begin{ldescription}
\item[\code{datasources}] A list of Opal object(s) as a handle to the server-side session

\item[\code{newobj}] The name of the server-side object where the parameters are stored

\item[\code{parts}] A vector with names for \code{TrainLayer} objects which have been created
by \code{\LinkA{ds.bm.defineLayer}{ds.bm.defineLayer}} before.
\end{ldescription}
\end{Arguments}
\inputencoding{utf8}
\HeaderA{ds.bm.exactloglikelihood}{Exact calculation of the log-likelihood of a Boltzmann machine}{ds.bm.exactloglikelihood}
%
\begin{Description}\relax
Calculates the log-likelihood of a Boltzmann machine model.
\end{Description}
%
\begin{Usage}
\begin{verbatim}
ds.bm.exactloglikelihood(datasources, bm, data = "D")
\end{verbatim}
\end{Usage}
%
\begin{Arguments}
\begin{ldescription}
\item[\code{datasources}] A list of Opal object(s) as a handle to the server-side session

\item[\code{bm}] The name of the Boltzmann machine model on the server-side.

\item[\code{data}] The name of the variable that holds the data on the server-side.
Defaults to \code{"D"}.
\end{ldescription}
\end{Arguments}
%
\begin{Details}\relax
\emph{This is only feasible for very small models, as the runtime grows exponentially with
the number of hidden nodes.}
\end{Details}
\inputencoding{utf8}
\HeaderA{ds.bm.samples}{Generate samples from a Boltzmann machine model}{ds.bm.samples}
%
\begin{Description}\relax
A Gibbs sampler is run in the Boltzmann machine model to sample from the learnt
distribution. This can also be used for sampling from a
\emph{conditional distribution}
(see arguments \code{conditionIndex} and \code{conditionValue} below.)
\end{Description}
%
\begin{Usage}
\begin{verbatim}
ds.bm.samples(
  datasources,
  bm,
  nsamples,
  burnin = NULL,
  conditionIndex = NULL,
  conditionValue = NULL,
  samplelast = NULL
)
\end{verbatim}
\end{Usage}
%
\begin{Arguments}
\begin{ldescription}
\item[\code{datasources}] A list of Opal object(s) as a handle to the server-side session

\item[\code{bm}] The name of the model to sample from on the server-side

\item[\code{nsamples}] The number of samples to generate

\item[\code{burnin}] The number of Gibbs sampling steps, defaults to 50.

\item[\code{conditionIndex}] A vector containing indices of variables that are to be conditioned on

\item[\code{conditionValue}] A vector containing the values for the variables that are to be conditioned on.
(must be of same length as \code{conditionIndex})

\item[\code{samplelast}] boolean to indicate whether to sample in the last step (\code{TRUE}, default)
or whether to use the deterministic activation potential.
\end{ldescription}
\end{Arguments}
\inputencoding{utf8}
\HeaderA{ds.dbm.exactloglikelihood}{Exact calculation of the log-likelihood of a DBM}{ds.dbm.exactloglikelihood}
%
\begin{Description}\relax
Same as \code{\LinkA{ds.bm.exactloglikelihood}{ds.bm.exactloglikelihood}}, only with parameter \code{bm} changed to \code{dbm}.
\end{Description}
%
\begin{Usage}
\begin{verbatim}
ds.dbm.exactloglikelihood(datasources, dbm = "dbm", data = "D")
\end{verbatim}
\end{Usage}
%
\begin{Arguments}
\begin{ldescription}
\item[\code{dbm}] The name of the model to sample from on the server-side. Defaults to \code{"dbm"}.
\end{ldescription}
\end{Arguments}
\inputencoding{utf8}
\HeaderA{ds.dbm.loglikelihood}{Likelihood estimation for a DBM model}{ds.dbm.loglikelihood}
%
\begin{Description}\relax
Estimates the log-likelihood of a DBM.
For this, separate runs of the annealed importance sampling algorithm (AIS)
are performed in addition to the esimation of the partition function of the DBM
via AIS.
\end{Description}
%
\begin{Usage}
\begin{verbatim}
ds.dbm.loglikelihood(
  datasources,
  dbm = "dbm",
  data = "D",
  parallelized = NULL,
  ntemperatures = NULL,
  nparticles = NULL,
  burnin = NULL
)
\end{verbatim}
\end{Usage}
%
\begin{Arguments}
\begin{ldescription}
\item[\code{datasources}] A list of Opal object(s) as a handle to the server-side session

\item[\code{dbm}] The name of the DBM model on the server-side. Defaults to \code{"dbm"}.

\item[\code{data}] The name of the variable that holds the data on the server-side.
Defaults to \code{"D"}.

\item[\code{ntemperatures}] The number of temperatures for annealing from the starting model
to the target model, defaults to 100

\item[\code{nparticles}] The number of parallel chains and calculated weights in AIS, defaults to 100

\item[\code{burnin}] The number of steps to sample for the Gibbs transition between the intermediate models in AIS
\end{ldescription}
\end{Arguments}
\inputencoding{utf8}
\HeaderA{ds.dbm.logproblowerbound}{Estimation of the variational lower bound of the log probability of a DBM model}{ds.dbm.logproblowerbound}
%
\begin{Description}\relax
Estimates the variational lower bound of the likelihood of a DBM using
annealed importance sampling (AIS).
\end{Description}
%
\begin{Usage}
\begin{verbatim}
ds.dbm.logproblowerbound(
  datasources,
  dbm = "dbm",
  data = "D",
  parallelized = NULL,
  ntemperatures = NULL,
  nparticles = NULL,
  burnin = NULL
)
\end{verbatim}
\end{Usage}
%
\begin{Arguments}
\begin{ldescription}
\item[\code{datasources}] A list of Opal object(s) as a handle to the server-side session

\item[\code{dbm}] The name of the DBM model on the server-side. Defaults to \code{"dbm"}.

\item[\code{data}] The name of the variable that holds the data on the server-side.
Defaults to \code{"D"}.

\item[\code{ntemperatures}] Number of temperatures for annealing from the starting model
to the target model, defaults to 100

\item[\code{nparticles}] Number of parallel chains and calculated weights in AIS, defaults to 100

\item[\code{burnin}] Number of steps to sample for the Gibbs transition between the intermediate models in AIS
\end{ldescription}
\end{Arguments}
\inputencoding{utf8}
\HeaderA{ds.dbm.samples}{Generate samples from a deep Boltzmann machine}{ds.dbm.samples}
%
\begin{Description}\relax
Same as \code{\LinkA{ds.bm.samples}{ds.bm.samples}}, only with parameter \code{bm} changed to \code{dbm}.
\end{Description}
%
\begin{Usage}
\begin{verbatim}
ds.dbm.samples(datasources, dbm = "dbm", ...)
\end{verbatim}
\end{Usage}
%
\begin{Arguments}
\begin{ldescription}
\item[\code{dbm}] The name of the model to sample from on the server-side. Defaults to \code{"dbm"}.
\end{ldescription}
\end{Arguments}
\inputencoding{utf8}
\HeaderA{ds.dbm.top2LatentDims}{Two-dimensional representation of latent features}{ds.dbm.top2LatentDims}
%
\begin{Description}\relax
Calculates the mean-field approximation of the activation of the top hidden nodes in
the DBM. If there are more than two nodes in the top layer, the dimensionality is further
reduced using a principal component analysis (PCA).
\end{Description}
%
\begin{Usage}
\begin{verbatim}
ds.dbm.top2LatentDims(datasources, dbm = "dbm", data = "D")
\end{verbatim}
\end{Usage}
%
\begin{Arguments}
\begin{ldescription}
\item[\code{datasources}] A list of Opal object(s) as a handle to the server-side session

\item[\code{dbm}] The name of the DBM model on the server-side. Defaults to \code{"dbm"}.

\item[\code{data}] The name of the variable that holds the data on the server-side.
Defaults to \code{"D"}.
\end{ldescription}
\end{Arguments}
%
\begin{Value}
A matrix with two columns for each of the sites,
containing the two-dimensional representation for
each of the samples in the rows. The samples are shuffled at random.
\end{Value}
\inputencoding{utf8}
\HeaderA{ds.monitored\_fitdbm}{Fits a (multimodal) DBM model}{ds.monitored.Rul.fitdbm}
%
\begin{Description}\relax
The procedure for DBM fitting consists of two parts:
First a stack of RBMs is pretrained in a greedy layerwise manner
(see \code{\LinkA{ds.monitored\_stackrbms}{ds.monitored.Rul.stackrbms}}). Then the weights of all layers are jointly
trained using the general Boltzmann machine learning procedure.
During pre-training and fine-tuning, monitoring data is collected by default.
The monitoring data is returned to the user.
The trained model is stored on the server side (see parameter \code{newobj}).
\end{Description}
%
\begin{Usage}
\begin{verbatim}
ds.monitored_fitdbm(
  datasources,
  newobj = "dbm",
  data = "D",
  monitoring = "logproblowerbound",
  monitoringdata = data,
  monitoringpretraining = "reconstructionerror",
  monitoringdatapretraining = monitoringdata,
  nhiddens = NULL,
  epochs = NULL,
  nparticles = NULL,
  learningrate = NULL,
  learningrates = NULL,
  learningratepretraining = NULL,
  epochspretraining = NULL,
  batchsizepretraining = NULL,
  pretraining = NULL
)
\end{verbatim}
\end{Usage}
%
\begin{Arguments}
\begin{ldescription}
\item[\code{datasources}] A list of Opal object(s) as a handle to the server-side session

\item[\code{newobj}] The name of the variable in which the trained DBM will be stored.
Defaults to \code{"dbm"}

\item[\code{data}] The name of the variable that holds the data on the server-side.
Defaults to \code{"D"}.

\item[\code{monitoring}] Name(s) for the monitoring options used for monitoring the fine-tuning.
Possible options:
\begin{itemize}

\item \code{"logproblowerbound"}: Variational lower bound of log probability (Default)
\item \code{"exactloglikelihood"}: Exact calculation of log-likelihood.
This is only feasible for very small models.
\item \code{NULL}: No monitoring

\end{itemize}


\item[\code{monitoringdata}] A vector of names for server-side data sets that are to be used for
monitoring

\item[\code{monitoringpretraining}] Name for monitoring options used for monitoring the pre-training.
The options are the same as for
training an RBM (see \code{\LinkA{ds.monitored\_fitrbm}{ds.monitored.Rul.fitrbm}}).
By default, the reconstruction error is monitored.

\item[\code{monitoringdatapretraining}] A vector of names for data sets that are to be used for
monitoring the pretraining. By default, this is the same as the \code{monitoringdata}.

\item[\code{nhiddens}] A vector that defines the number of nodes in the hidden layers of
the DBM. The default value specifies two hidden layers with the same size
as the visible layer.

\item[\code{epochs}] Number of training epochs for fine-tuning, defaults to 10

\item[\code{nparticles}] Number of particles used for sampling during fine-tuning of the
DBM, defaults to 100

\item[\code{learningrate}] Learning rate for joint training of layers (= fine-tuning)
using the learning algorithm for a general Boltzmann machine
with mean-field approximation.
The learning rate for fine tuning is by default decaying with the number of epochs,
starting with the given value for the \code{learningrate}.
By default, the learning rate decreases with the factor \eqn{11 / (10 + epoch)}{}.

\item[\code{learningrates}] A vector of learning rates for each epoch of fine-tuning

\item[\code{learningratepretraining}] Learning rate for pretraining,
defaults to \code{learningrate}

\item[\code{epochspretraining}] Number of training epochs for pretraining,
defaults to \code{epochs}

\item[\code{batchsizepretraining}] Batchsize for pretraining, defaults to 1

\item[\code{pretraining}] The arguments for layerwise pretraining
can be specified for each layer individually.
This is done via a vector of names for objects that have previously been defined
by \code{\LinkA{ds.bm.defineLayer}{ds.bm.defineLayer}} or \code{\LinkA{ds.bm.definePartitionedLayer}{ds.bm.definePartitionedLayer}}.
(For a detailed description of the possible parameters,
see the help there).
If the number of training epochs and the learning rate are not specified
explicitly for a layer, the values of \code{epochspretraining},
\code{learningratepretraining} and \code{batchsizepretraining} are used.
\end{ldescription}
\end{Arguments}
%
\begin{Details}\relax
If the option \code{dsBoltzmannMachines.shareModels} is set to \code{TRUE}
by an administrator at the server side, the models themselves are returned in addition.
\end{Details}
\inputencoding{utf8}
\HeaderA{ds.monitored\_fitrbm}{Fit an RBM model}{ds.monitored.Rul.fitrbm}
%
\begin{Description}\relax
Fits an RBM model using Stochastic Gradient Descent (SGD) on the \code{data}
with Contrastive Divergence (CD).
During the training, monitoring data is collected by default.
The monitoring data is returned to the user.
The trained model is stored on the server side (see parameter \code{newobj}).
\end{Description}
%
\begin{Usage}
\begin{verbatim}
ds.monitored_fitrbm(
  datasources,
  data = "D",
  newobj = "rbm",
  monitoring = "reconstructionerror",
  monitoringdata = NULL,
  nhidden = NULL,
  epochs = NULL,
  upfactor = NULL,
  downfactor = NULL,
  learningrate = NULL,
  learningrates = NULL,
  pcd = NULL,
  cdsteps = NULL,
  categories = NULL,
  batchsize = NULL,
  rbmtype = NULL,
  startrbm = NULL
)
\end{verbatim}
\end{Usage}
%
\begin{Arguments}
\begin{ldescription}
\item[\code{datasources}] A list of Opal object(s) as a handle to the server-side session

\item[\code{data}] The name of the variable that holds the training data on the server-side.
Defaults to \code{"D"}.

\item[\code{newobj}] The name for the variable in which the trained RBM will be stored.
Defaults to \code{"rbm"}.

\item[\code{monitoring}] Name(s) for monitoring options used for RBM training.
Possible options:
\begin{itemize}

\item \code{"reconstructionerror"}: Calculates the reconstruction error (Default)
\item \code{"loglikelihood"}: Estimates the loglikelihood via annealed importance sampling (AIS)
\item \code{"exactloglikelihood"}: Exact calculation of log-likelihood.
This is only feasible for very small models.
\item \code{NULL}: No monitoring

\end{itemize}


\item[\code{monitoringdata}] A vector of names for server-side data sets that are to be used for
monitoring

\item[\code{nhidden}] The number of hidden units of the returned RBM

\item[\code{epochs}] The number of training epochs

\item[\code{upfactor}] If this function is used for pretraining a part of
a DBM, it is necessary to multiply the input from the visible layer of the RBM with a factor.

\item[\code{downfactor}] If this function is used for pretraining a part of
a DBM, it is necessary to multiply the input from the hidden layer of the RBM with a factor.

\item[\code{learningrate}] The learning rate for the weights and biases
can be specified as a single value, used throughout all epochs. Defaults to 0.005.

\item[\code{learningrates}] The learning rate for the weights and biases
can also be specified as a vector that contains a value for each epoch.

\item[\code{pcd}] Indicating whether Persistent Contrastive Divergence (PCD) is to
be used (\code{TRUE}, default) or simple CD that initializes the Gibbs chain with
the training sample (\code{FALSE})

\item[\code{cdsteps}] The number of Gibbs sampling steps for (persistent)
contrastive divergence. Defaults to 1.

\item[\code{categories}] Only relevant if \code{rbmtype} is \code{"Softmax0BernoulliRBM"}.
The number of categories, if all variables have the same number
of categories, or a vector that contains the number of categories
for the i'th categorical variable in the i'th entry.

\item[\code{batchsize}] The number of samples that are used for making one step in the
stochastic gradient descent optimizer algorithm. Default is 1.

\item[\code{rbmtype}] The type of the RBM that is to be trained.
This must be a subtype of \code{AbstractRBM}. Defaults to \code{BernoulliRBM}.

\item[\code{startrbm}] A name for an RBM object at the server side that
is used as starting value for training.
If this argument is specified, \code{nhidden} and \code{rbmtype} are ignored.
\end{ldescription}
\end{Arguments}
%
\begin{Details}\relax
If the option \code{dsBoltzmannMachines.shareModels} is set to \code{TRUE}
by an administrator at the server side, the models themselves are returned in addition.
\end{Details}
\inputencoding{utf8}
\HeaderA{ds.monitored\_stackrbms}{Train a stack of RBMs}{ds.monitored.Rul.stackrbms}
%
\begin{Description}\relax
Performs greedy layerwise training for deep belief networks or greedy layerwise
pretraining for deep Boltzmann machines.
During the training, monitoring data is collected by default.
The monitoring data is returned to the user.
The trained model is stored on the server side (see parameter \code{newobj}).
\end{Description}
%
\begin{Usage}
\begin{verbatim}
ds.monitored_stackrbms(
  datasources,
  data = "D",
  newobj = "rbmstack",
  monitoring = "reconstructionerror",
  monitoringdata = NULL,
  nhiddens = NULL,
  epochs = NULL,
  predbm = NULL,
  samplehidden = NULL,
  learningrate = NULL,
  batchsize = NULL,
  trainlayers = NULL
)
\end{verbatim}
\end{Usage}
%
\begin{Arguments}
\begin{ldescription}
\item[\code{datasources}] A list of Opal object(s) as a handle to the server-side session

\item[\code{data}] The name of the variable that holds the data on the server-side.
Defaults to \code{"D"}.

\item[\code{newobj}] The name of the variable in which the trained RBM will be stored.
Defaults to \code{"rbmstack"}.

\item[\code{monitoring}] Name(s) for monitoring options used for RBM training.
For possible options, see \code{\LinkA{ds.monitored\_fitrbm}{ds.monitored.Rul.fitrbm}}

\item[\code{monitoringdata}] A vector of names for server-side data sets that are to be used for
monitoring. The data is propagated forward through the
network to monitor higher levels.

\item[\code{nhiddens}] A vector containing the number of nodes of the i'th hidden layer in
the i'th entry

\item[\code{epochs}] The number of training epochs

\item[\code{predbm}] logical value indicating that the greedy layerwise training is
pre-training for a DBM.
If its value is \code{FALSE} (default), a DBN is trained.

\item[\code{samplehidden}] logical value indicating that consequent layers are to be trained
with sampled values instead of the deterministic potential.
Using the deterministic potential (\code{FALSE}) is the default.

\item[\code{learningrate}] The learningrate used for training the RBMs. Defaults to 0.005.

\item[\code{batchsize}] The size of the training minibatches. Defaults to 1.

\item[\code{trainlayers}] A vector of names of \code{TrainLayer} objects.
With this argument it is possible
to specify the training parameters for each layer/RBM individually.
If the number of training epochs and the learning rate are not specified
explicitly for a layer, the values of \code{epochs}, \code{learningrate}
and \code{batchsize} are used.
For more information see help of \code{\LinkA{ds.bm.defineLayer}{ds.bm.defineLayer}}.
\end{ldescription}
\end{Arguments}
%
\begin{Details}\relax
If the option \code{dsBoltzmannMachines.shareModels} is set to \code{TRUE}
by an administrator at the server side, the model itself is returned in addition.
\end{Details}
\inputencoding{utf8}
\HeaderA{ds.monitored\_traindbm}{Fine-Tuning of a DBM}{ds.monitored.Rul.traindbm}
%
\begin{Description}\relax
This functions performs monitored fine-tuning of a given DBM model.
For the complete training, including the pre-training, see \code{\LinkA{ds.monitored\_fitdbm}{ds.monitored.Rul.fitdbm}}.
During the training, monitoring data is collected by default.
The monitoring data is returned to the user.
The trained model is stored on the server side (see parameter \code{newobj}).
\end{Description}
%
\begin{Usage}
\begin{verbatim}
ds.monitored_traindbm(
  datasources,
  dbm = "dbm",
  newobj = "dbm",
  data = "D",
  monitoring = "logproblowerbound",
  monitoringdata = data,
  epochs = NULL,
  nparticles = NULL,
  learningrate = NULL,
  learningrates = NULL
)
\end{verbatim}
\end{Usage}
%
\begin{Arguments}
\begin{ldescription}
\item[\code{dbm}] The name of DBM model that is to be fine-tuned. Defaults to \code{"dbm"}.

\item[\code{newobj}] The name of the variable to store the new DBM model.
Defaults to \code{"dbm"}, such that the previous model is overwritten.

\item[\code{monitoring}] Name(s) for monitoring options used for DBM training.
For possible options, see \code{\LinkA{ds.monitored\_fitdbm}{ds.monitored.Rul.fitdbm}}

\item[\code{monitoringdata}] A vector of names of server-side data sets that are to be used for
monitoring

\item[\code{epochs}] Number of training epochs for fine-tuning, defaults to 10

\item[\code{nparticles}] Number of particles used for sampling during fine-tuning of the
DBM, defaults to 100

\item[\code{learningrate}] Learning rate for fine-tuning,
decaying by default decaying with the number of epochs,
starting with the given value for the \code{learningrate}.
By default, the learning rate decreases with the factor \eqn{11 / (10 + epoch)}{}.

\item[\code{learningrates}] A vector of learning rates for each epoch of fine-tuning
\end{ldescription}
\end{Arguments}
%
\begin{Details}\relax
If the option \code{dsBoltzmannMachines.shareModels} is set to \code{TRUE}
by an administrator at the server side, the model itself is returned in addition.
\end{Details}
\inputencoding{utf8}
\HeaderA{ds.rbm.exactloglikelihood}{Exact calculation of the log-likelihood of an RBM}{ds.rbm.exactloglikelihood}
%
\begin{Description}\relax
Same as \code{\LinkA{ds.bm.exactloglikelihood}{ds.bm.exactloglikelihood}}, only with parameter \code{bm} changed to \code{rbm}.
\end{Description}
%
\begin{Usage}
\begin{verbatim}
ds.rbm.exactloglikelihood(datasources, rbm = "rbm", data = "D")
\end{verbatim}
\end{Usage}
%
\begin{Arguments}
\begin{ldescription}
\item[\code{rbm}] The name of the model to sample from on the server-side. Defaults to \code{"rbm"}.
\end{ldescription}
\end{Arguments}
\inputencoding{utf8}
\HeaderA{ds.rbm.loglikelihood}{Likelihood  estimation for an RBM model}{ds.rbm.loglikelihood}
%
\begin{Description}\relax
Estimates the log-likelihood for an RBM model using annealed importance sampling (AIS)
for estimating the partition function.
\end{Description}
%
\begin{Usage}
\begin{verbatim}
ds.rbm.loglikelihood(
  datasources,
  rbm = "rbm",
  data = "D",
  parallelized = NULL,
  ntemperatures = NULL,
  nparticles = NULL,
  burnin = NULL
)
\end{verbatim}
\end{Usage}
%
\begin{Arguments}
\begin{ldescription}
\item[\code{datasources}] A list of Opal object(s) as a handle to the server-side session

\item[\code{rbm}] The name of the RBM model on the server. Defaults to \code{"rbm"}.

\item[\code{data}] The name of the variable that holds the data on the server-side.
Defaults to \code{"D"}.

\item[\code{ntemperatures}] Number of temperatures for annealing from the starting model
to the target model, defaults to 100

\item[\code{nparticles}] Number of parallel chains and calculated weights in AIS, defaults to 100

\item[\code{burnin}] Number of steps to sample for the Gibbs transition between the intermediate models in AIS
\end{ldescription}
\end{Arguments}
\inputencoding{utf8}
\HeaderA{ds.rbm.samples}{Generate samples from a restricted Boltzmann machine}{ds.rbm.samples}
%
\begin{Description}\relax
Same as \code{\LinkA{ds.bm.samples}{ds.bm.samples}}, only with parameter \code{bm} changed to \code{rbm}.
\end{Description}
%
\begin{Usage}
\begin{verbatim}
ds.rbm.samples(datasources, rbm = "rbm", ...)
\end{verbatim}
\end{Usage}
%
\begin{Arguments}
\begin{ldescription}
\item[\code{rbm}] The name of the model to sample from on the server-side. Defaults to \code{"rbm"}.
\end{ldescription}
\end{Arguments}
\inputencoding{utf8}
\HeaderA{ds.setJuliaSeed}{Set a random seed in Julia}{ds.setJuliaSeed}
%
\begin{Description}\relax
Set a seed for the random generator in Julia.
This makes calls to the non-deterministic algorithms in this package reproducible.
\end{Description}
%
\begin{Usage}
\begin{verbatim}
ds.setJuliaSeed(datasources, seed)
\end{verbatim}
\end{Usage}
%
\begin{Arguments}
\begin{ldescription}
\item[\code{seed}] A single integer value that is used as the seed for the random number generator in Julia
\end{ldescription}
\end{Arguments}
\inputencoding{utf8}
\HeaderA{ds.splitdata}{Split samples of a data set}{ds.splitdata}
%
\begin{Description}\relax
Splits a data set randomly into two data sets \eqn{x_1}{} and \eqn{x_2}{}, such that
the fraction of samples in \eqn{x_2}{} is equal to (or as close as possible to) the
given \code{ratio}.
\end{Description}
%
\begin{Usage}
\begin{verbatim}
ds.splitdata(datasources, data, ratio, newobj1, newobj2)
\end{verbatim}
\end{Usage}
%
\begin{Arguments}
\begin{ldescription}
\item[\code{datasources}] A list of Opal object(s) as a handle to the server-side session

\item[\code{data}] The name of the variable that holds the data on the server-side. Defaults to \code{"D"}.

\item[\code{ratio}] The ratio of samples

\item[\code{newobj1}] The name for the variable where \eqn{x_1}{} is stored

\item[\code{newobj2}] The name for the variable where \eqn{x_2}{} is stored
\end{ldescription}
\end{Arguments}
%
\begin{Examples}
\begin{ExampleCode}
    ds.splitdata(o, "D", 0.1, "D.Train", "D.Test")
\end{ExampleCode}
\end{Examples}
